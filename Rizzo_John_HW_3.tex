\documentclass[12pt, letterpaper]{article}
% Custom Commands for User Info
\newcommand{\studentname}          {John Rizzo           }
\newcommand{\classname}            {CS590-A Algorithms   }
\newcommand{\professorname}        {Dr. William Hendrix  }
\newcommand{\assignmentdescription}{Homework 3 Algorithms}
\newcommand{\duedate}              {February 27, 2025     }

\title{\classname \\ \assignmentdescription}
\author{\studentname}
\date{\duedate}

% Packages
\usepackage[utf8]{inputenc}
\usepackage{amsmath}
\usepackage{amssymb}
\usepackage{geometry}
\geometry{margin=1in}
\usepackage{fancyhdr}
\usepackage{datetime}
\usepackage{blindtext}
\usepackage{pgfplots}
\usepackage{minted}
% \usepackage{algorithm}
% \usepackage{algorithmic}
% \usepackage{algpseudocode}

% \usepackage[boxed,linesnumbered,vlined]{algorithm2e}
% \SetStartEndCondition{ }{}{}%
% \SetKwProg{Fn}{def}{\string:}{}
% \SetKwProg{Class}{class}{\string:}{}
% \SetKwFunction{Range}{range}%%
% \SetKw{KwTo}{in}\SetKwFor{For}{for}{\string:}{}%
% \SetKwIF{If}{ElseIf}{Else}{if}{:}{elif}{else:}{}%
% \SetKwFor{While}{while}{:}{fintq}%
% \AlgoDontDisplayBlockMarkers\SetAlgoNoEnd\SetAlgoNoLine%
% \DontPrintSemicolon
% \renewcommand{\forcond}{$i$ \KwTo\Range{$n$}}

% Header and Footer setup
\renewcommand{\headrulewidth}{0.4pt}
\renewcommand{\footrulewidth}{0.4pt}
\setlength{\headheight}{14.49998pt}
\addtolength{\topmargin}{-2.49998pt}

\pgfplotsset{width=10cm,compat=1.18}

% Document Start
\begin{document}

\noindent
\normalsize \textbf{Student:     \studentname} \\ [5pt]
            \textbf{Course:      \classname} \\ [5pt]
            \textbf{Instructor:  \professorname} \\ [5pt]
            \textbf{Due Date:    \duedate} \\ [5pt]
            \textbf{Description: \assignmentdescription}

\vspace{0.5cm}

% Problem Sections
\section*{Problem 1}

What new field(s) does the data structure need?

\vspace{0.5cm}
\noindent
The new solution requires that the root node is augmented to store the minimum value, such as in node.minval.
\section*{Problem 2}

Give pseudo code for the min operation for the BST.

% \TitleOfAlgo{BST.min()}
% \KwOut{The minimum value in the tree}
% \begin{algorithm}[H]
%     \Fn{BST.min()}{
%         node = root \\
%         \If{node $\neq$ NIL}{
%             \Return node.minval
%         }
%     }
% \end{algorithm}

\begin{minted}{python}
    def BST.min():
        node = root
        if node != None:
            return node.minval
\end{minted}

% \break
\section*{Problem 3}

Give pseudo code for the insert operation.  Reference pseudo code for the insert method appears below.

\begin{minted}{python}
    def BST.insert(newvalue, root):
        node = root
        while node != None:
            if node.value <= newvalue:
                if root.left == None:
                    root.left = node
                node = node.left
                if root.minval > node.value:
                    root.minval = node.value
            else:
                if root.right = None:
                    node = node.right
                node = node.right
\end{minted}

\break
\section*{Problem 4}

\begin{minted}{python}
    def BST.delete(node):
        if node.left != None and node.right != None:
            swapnode = right
            while swapnode.left != None:
                swapnode = swapnode.left
            Swap node's parent and children links with swapnode
            if node is self.root:
                root = swapnode
        if node.left == None and node.right == None:
            if node == self.root:
                root = None
            else:
                node.parent.left = None
                node.parent.right = None
        else:
            # Node must have one child
            if node == self.root:
                Set root to be node's child
            else:
                set node.parent's child to be node's child
            Set node's child's parent to be node.parent
            Find the minimum value from the root node
            set the root's min to be the minimum value
\end{minted}

\break
\section*{Problem 5}

Give pseudo code for an efficient algorithm for the \textit{ top k} search problem.
In top-k search, you are given an array of $n$ integers and must return the $k$
largest integers, where $k$ is generally much smaller than $n$.  Acceptable
algorithms might be $O(n+klgn)$ or $O(nlgk)$, but not $O(nk)$ or $O(nlgn)$.
\textit{Hint} use an appropriate data structure!

\vspace{0.5em}

\noindent
Given the following heap data structure, we can implement a top-k search algorithm, which is $O(n+klgn)$.

\begin{minted}{python}
    class MaxHeap:
        def __init__(self):
            self.heap = []

        def insert(self, element):
            self.heap.append(element)
            self._heapify_up(len(self.heap) - 1)

        def max(self):
            if not self.heap:
                return None
            return self.heap[0]

        def extract_max(self):
            if len(self.heap) == 0:
                return None
            if len(self.heap) == 1: return self.heap.pop()
            root = self.heap[0]
            self.heap[0] = self.heap.pop()
            self._heapify_down(0)
            return root

        def _heapify_up(self, index):
            parent_index = (index - 1) // 2
            if index > 0 and self.heap[index] > self.heap[parent_index]:
                self.heap[index], self.heap[parent_index] = 
                    self.heap[parent_index], self.heap[index]
                self._heapify_up(parent_index)

        def _heapify_down(self, index):
            largest = index
            left_child_index = 2 * index + 1
            right_child_index = 2 * index + 2

            if left_child_index < len(self.heap) and 
                self.heap[left_child_index] > self.heap[largest]:
                    largest = left_child_index

            if right_child_index < len(self.heap) and
                self.heap[right_child_index] > self.heap[largest]:
                    largest = right_child_index

            if largest != index:
                self.heap[index], self.heap[largest] =
                    self.heap[largest], self.heap[index]
                self._heapify_down(largest)
\end{minted}


\begin{minted}{python}
    def topk(arr):
        heap = 0
        result = 0
        
        for i = 0 to n:
            heap.insert(arr[i])
        
        for i = 1 to (k+1):
            max = heap.max()
            heap.delete(max)
            result.insert(max)
        
        return result
\end{minted}

% \begin{algorithm}
%     \caption{top-k Search}
%     \begin{minipage}{\textwidth}
%         \begin{algorithmic}[1]
%             \STATE $heap = \emptyset$
%             \STATE $result = \emptyset$
%             \FOR{$i = 0$ to $n$}
%                 \STATE Insert $arr[i]$ into $heap$
%             \ENDFOR
%             \FOR{$i = 1$ to $k+1$}
%                 \STATE $max = heap.max()$
%                 \STATE $heap.delete(max)$
%                 \STATE $result.insert(max)$
%             \ENDFOR
%             \RETURN $result$
%         \end{algorithmic}
%     \end{minipage}
% \end{algorithm}

\end{document}