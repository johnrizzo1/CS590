\documentclass[12pt, letterpaper]{article}
% Custom Commands for User Info
\newcommand{\studentname}          {John Rizzo           }
\newcommand{\classname}            {CS590-A Algorithms   }
\newcommand{\professorname}        {Dr. William Hendrix  }
\newcommand{\assignmentdescription}{Homework 3 Algorithms}
\newcommand{\duedate}              {February 27, 2025     }

\title{\classname \\ \assignmentdescription}
\author{\studentname}
\date{\duedate}

% Packages
\usepackage[utf8]{inputenc}
\usepackage{amsmath}
\usepackage{amssymb}
\usepackage{geometry}
\geometry{margin=1in}
\usepackage{fancyhdr}
\usepackage{datetime}

\usepackage{pgfplots}

\usepackage{algorithm}
\usepackage{algorithmic}
% \usepackage{algpseudocode}

\usepackage{blindtext}

% Header and Footer setup
\renewcommand{\headrulewidth}{0.4pt}
\renewcommand{\footrulewidth}{0.4pt}
\setlength{\headheight}{14.49998pt}
\addtolength{\topmargin}{-2.49998pt}

\pgfplotsset{width=10cm,compat=1.18}

% Document Start
\begin{document}

\noindent
\normalsize \textbf{Student:     \studentname} \\ [5pt]
            \textbf{Course:      \classname} \\ [5pt]
            \textbf{Instructor:  \professorname} \\ [5pt]
            \textbf{Due Date:    \duedate} \\ [5pt]
            \textbf{Description: \assignmentdescription}

\vspace{0.5cm}

% Problem Sections
\section*{Problem 1}

What new field(s) does the data structure need?

\vspace{0.5cm}
\noindent
The new solution requires that the root node is augmented to store the minimum value, such as in node.minval.

\section*{Problem 2}

Give pseudocode for the min operation for the BST.
\begin{algorithm}
    \caption{BST.min()}
    \begin{minipage}{\textwidth}
        \textbf{Output}: The minimum value in the tree
        \begin{algorithmic}[1]
            \STATE node = root
            \IF{node $\neq$ NIL}
                \RETURN node.minval
            \ENDIF
        \end{algorithmic}
    \end{minipage}
\end{algorithm}

\break
\section*{Problem 3}

Give pseudocode for the insert operation.  Reference pseudocode for the insert method appears below.

\begin{algorithm}
    \caption{BST.insert()}
    \begin{minipage}{\textwidth}
        \begin{algorithmic}[1]
            \STATE $node = root$
            \WHILE{$node \neq NIL$}
                \IF{$node.value \leq new$}
                    \IF{$node.left = NIL$}
                        \STATE Add $new$ as left child of $node$ \\
                        \STATE $node = node.left$
                    \ELSE
                        \STATE $node = node.left$
                    \ENDIF
                    \IF{$root.minval > node.value$}
                        \STATE $root.minval = node.value$
                    \ENDIF
                \ELSE
                    \IF{$node.right = NIL$}
                        \STATE Add $new$ as right child of $node$ \\
                        $node = node.right$
                    \ELSE
                        \STATE $node = node.right$
                    \ENDIF
                \ENDIF
            \ENDWHILE
        \end{algorithmic}
    \end{minipage}
\end{algorithm}

\break
\section*{Problem 4}

\begin{algorithm}
    \caption{BST.delete(node)}
    \begin{minipage}{\textwidth}
        \begin{algorithmic}[1]
            \IF{$node$ has two children}
                \STATE $swapnode = right$
                \WHILE{$swapnode$ has a left $child$}
                    \STATE $swapnode = swapnode.left$
                \ENDWHILE
                \STATE Swap node's parent and children links with $swapnode$
                \IF{$node$ is the BST $root$}
                    \STATE Set root to be $swapnode$
                \ENDIF
            \ENDIF
            \IF{$node$ has no children}
                \IF{$node$ is the root}
                    \STATE Set root to be NIL
                \ELSE
                    \STATE Set node.parent's child to be NIL
                \ENDIF
            \ELSE
                \STATE // $node$ must have one child
                \IF{$node$ is the root}
                    \STATE Set root to be node's child
                \ELSE
                    \STATE Set node.parent's child to be node's child
                \ENDIF
                \STATE Set node's child's parent to be node.parent
                \STATE Find the minimum value from the root
                \STATE Set the root's min to be the minimum value
            \ENDIF
        \end{algorithmic}
    \end{minipage}
\end{algorithm}

\break
\section*{Problem 5}

Give pseudocode for an eficient algorithm for the \textit{top-k} search problem.
In top-k search, you are given an array of $n$ integers and must return the $k$
largest integers, where $k$ is generally much smaller than $n$.  Acceptable
algorithms might be $O(n+klgn)$ or $O(nlgk)$, but not $O(nk)$ or $O(nlgn)$.
\textit{Hint} use an appropriate data structure!

\begin{algorithm}
    \caption{top-k Search}
    \begin{minipage}{\textwidth}
        \begin{algorithmic}[1]
            \STATE $heap = \emptyset$
            \STATE $result = \emptyset$
            \FOR{$i = 0$ to $n$}
                \STATE Insert $arr[i]$ into $heap$
            \ENDFOR
            \FOR{$i = 1$ to $k+1$}
                \STATE $max = heap.max()$
                \STATE $heap.delete(max)$
                \STATE $result.insert(max)$
            \ENDFOR
            \RETURN $result$
        \end{algorithmic}
    \end{minipage}
\end{algorithm}

\end{document}

