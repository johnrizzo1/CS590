\documentclass[12pt, letterpaper]{article}
% Custom Commands for User Info
\newcommand{\studentname}          {John Rizzo           }
\newcommand{\classname}            {CS590-A Algorithms   }
\newcommand{\professorname}        {Dr. William Hendrix  }
\newcommand{\assignmentdescription}{Homework 1 Algorithms}
\newcommand{\duedate}              {January 22, 2025     }

\title{\classname \\ \assignmentdescription}
\author{\studentname}
\date{\duedate}

% Packages
\usepackage[utf8]{inputenc}
\usepackage{amsmath}
\usepackage{amssymb}
\usepackage{geometry}
\geometry{margin=1in}
\usepackage{fancyhdr}
\usepackage{datetime}

% Header and Footer setup
% \pagestyle{fancy}
% \fancyhf{}
\renewcommand{\headrulewidth}{0.4pt}
\renewcommand{\footrulewidth}{0.4pt}
\setlength{\headheight}{14.49998pt}
\addtolength{\topmargin}{-2.49998pt}
% \rhead{\thepage}

% Document Start
\begin{document}

% Title Section
% \begin{center}
%     \Large \textbf{Homework Template} \\ [10pt]
% \end{center}
% \vspace{0.5cm}

\noindent
\normalsize \textbf{Student:     \studentname} \\ [5pt]
            \textbf{Course:      \classname} \\ [5pt]
            \textbf{Instructor:  \professorname} \\ [5pt]
            \textbf{Due Date:    \duedate} \\ [5pt]
            \textbf{Description: \assignmentdescription}

\vspace{0.5cm}

% Problem Sections
\section*{Problem 1}
\textbf{Definition 1.} $a \mid b$ \textit{("a divides b") if and only if there exists some 
integer $k$ such that $b = ak$. Equivalently, $a \mid b$ if and only if $b$ has a 
remainder of $0$ when divided by $a$ (see question 2).}

\begin{enumerate}
    \item Using the formal definition of divisibility above, prove that there exists positive integers $a$, $b$, and $c$ such that $a \mid bc$, but $a \nmid b$ and $a \nmid c$. \\ \\ \textbf{Theorem 1}. \textit{The Division Algorithm. For any integers $a$ and $b$ where $b \neq 0$, there exist a unique pair of integers $q$ and $r$ such that $a = qb + r$ and $0 \leq r \leq b$.  The integers $q$ and $r$ are known as the quotient and remainder of $a \div b$, respectively.}

    \item Using the formal definition of the remainder above, prove that if $n$ and $m$ are positive integers such that $n$ has a remainder of $r$ when divided by $m$ and $r < \sqrt{m}$, $n^2$ has a remainder of $r^2$ when divided by $m$.

    \item Use the formal definition of Big-Oh to prove that if $f(n) = n^x + an^y$, where $a$, $x$, and $y$ are positive integers such that $x > y$, $f(n) = O(n^x)$.

    \item Use the formal definition of Big-Omega to prove that if $f_1(n)$, $f_2(n)$, $g_1(n)$, and $g_2(n)$ are functions such that $f_1(n) = \Omega(g_1(n))$ and $f_2(n) = \Omega(g_2(n))$, $f_1(n) + f_2(n) = \Omega(\max(g_1(n), g_2(n)))$.
\end{enumerate}


\end{document}
