\documentclass[12pt, letterpaper]{article}
% Custom Commands for User Info
\newcommand{\studentname}          {John Rizzo           }
\newcommand{\classname}            {CS590-A Algorithms   }
\newcommand{\professorname}        {Dr. William Hendrix  }
\newcommand{\assignmentdescription}{Homework 2 Algorithms}
\newcommand{\duedate}              {February 17, 2025     }

\title{\classname \\ \assignmentdescription}
\author{\studentname}
\date{\duedate}

% Packages
\usepackage[utf8]{inputenc}
\usepackage{amsmath}
\usepackage{amssymb}
\usepackage{geometry}
\geometry{margin=1in}
\usepackage{fancyhdr}
\usepackage{datetime}
\usepackage{pgfplots}

% Header and Footer setup
\renewcommand{\headrulewidth}{0.4pt}
\renewcommand{\footrulewidth}{0.4pt}
\setlength{\headheight}{14.49998pt}
\addtolength{\topmargin}{-2.49998pt}

\pgfplotsset{width=10cm,compat=1.18}

% Document Start
\begin{document}

\noindent
\normalsize \textbf{Student:     \studentname} \\ [5pt]
            \textbf{Course:      \classname} \\ [5pt]
            \textbf{Instructor:  \professorname} \\ [5pt]
            \textbf{Due Date:    \duedate} \\ [5pt]
            \textbf{Description: \assignmentdescription}

\vspace{0.5cm}

% Problem Sections
\section*{Problem 1}

Analyze the \textit{worst-case} time complexity of the LoopMystery algorithm below.
Please show all work.  The $\lfloor \rfloor$ symbols represent the \textit{floor}
("round down") function. You may assume that this function takes $\Theta(1)$ time for any
input.  You may also assume it takes a constant amount of time to determine whether an
integer is odd.

\vspace{0.5cm}

\noindent
Note that figuring out what problem this algorithm solves is \textit{irrelevant} to analyzing its complexity.

\vspace{0.5cm}

\fbox {
    \begin{minipage}{\textwidth}
        \paragraph{Input}: n: nonnegative integer \\
        1 \textbf{Algorithm}: LoopMystery \\
        2 \textit{sum} = 0 \\
        3 \textit{$t$} = 1 \\
        4 \textit{$d$} = 1 \\
        5 \textit{$k$} = $n$ \\
        6 \textbf{while} \textit{$k$} $>$ 1 \textbf{do} \\
        7   \hspace*{4mm} \textbf{for} $i = 1$ to $k$ \textbf{do} \\
        8       \hspace*{8mm} $t = t + d$ \\
        9       \hspace*{8mm} $sum = sum + t$ \\
        10  \hspace*{4mm} \textbf{end} \\
        11  \hspace*{4mm} \textbf{if} $k$ is \textit{odd} \textbf{then} \\
        12      \hspace*{8mm} $d = -d$ \\
        13  \hspace*{4mm} \textbf{end} \\
        14  \hspace*{4mm} $k = \lfloor k/2 \rfloor$ \\
        15 \textbf{end} \\
        16 \textbf{return} \textit{sum} \\
    \end{minipage}
}

\begin{enumerate}
    \item lines 1-5 are simple constant time assignment operations
    \item lines 7-10 is the first inner loop with lines 8, 9 being constant time and since
        $k = \lfloor k/2 \rfloor$ using the ... approach if you choose you can see that
        
        $(n - 1) + \frac{(n-1)}{2} + \frac{(n-1)}{4}$.
        
        $(n - 1) ( \frac{1}{2} + \frac{1}{4} )$
        
        For large numbers of $n$ it will dominate the fraction.  Based on this sequence you can see that this 
        is a sum of iterations that matches $T(n) = \sum_{i=1}^{n}{\frac{1}{2^i}}$ = $\Theta(1)$
    \item lines 11-13 is a simple conditional which should be constant time.
    \item lines 6-16 is the largest outter loop will be dominated by the first inner loop
\end{enumerate}

\textbf{Answer} $T(n) = \sum_{i=1}^{n}{\frac{1}{2^i}}$ = $\Theta(1)$

\section*{Problem 2}

Find a recurrence $T(n)$ that describes the runtime of the RecursionMystery algorithm below:

\vspace{0.5cm}

\fbox {
    \begin{minipage}{\textwidth}
        \paragraph{Input}: data: array of integers
        \paragraph{Input}: n: size of data \\
        01 \textbf{Algorithm}: RecursionMystery \\
        02 \textbf{if} $n>1$ \textbf{then} \\
        03  \hspace*{4mm} $\textit{min} = \textit{max} = 1$ \\
        04  \hspace*{4mm} \textbf{for} $i=2$ to $n$ \textbf{do} \\
        05      \hspace*{8mm} \textbf{if} $data[i] < data[min]$ \textbf{then} \\
        06          \hspace*{12mm} $min = i$ \\
        07      \hspace*{8mm} \textbf{end} \\
        08      \hspace*{8mm} \textbf{if} $data[i] > data[max]$ \textbf{then} \\
        09          \hspace*{12mm} $max = i$ \\
        10      \hspace*{8mm} \textbf{end} \\
        11  \hspace*{4mm} \textbf{end} \\
        12  \hspace*{4mm} Swap $data[1]$ and $data[min]$ \\
        13  \hspace*{4mm} \textbf{if} $max > 1$ \textbf{then} \\
        14      \hspace*{8mm} Swap $data[n]$ and $data[max]$ \\
        15  \hspace*{4mm} \textbf{else} \\
        16      \hspace*{8mm} Swap $data[min]$ and $data[max]$ \\
        17  \hspace*{4mm} \textbf{end} \\
        18  \hspace*{4mm} \textbf{if} $n>2$ \textbf{then} \\
        19      \hspace*{8mm} Call RecursionMystery on $data[2..n-1]$ \\
        20  \hspace*{4mm} \textbf{end} \\
        21 \textbf{end} \\
        22 \textbf{return} \textit{data}
    \end{minipage}
}

\begin{enumerate}
    \item lines 1-3,12-17,21-22 are constant time
    \item lines 4-11 is a non-recursive loop which will occur a max of $n-1$ times
    \item lines 19 has the only recursive call which will occur a maximum of n-2 times. For example if $n=6$ then \\
        2..6-1, 2..6-2, 2..6-3 and so on.  This matches $\sum_{i=l}^{n}{i} = \Theta(x^2)$
\end{enumerate}

\textbf{Answer} $T(n) = \Theta(x^2)$

\section*{Problem 3}

Sketch the recurrence tree that corresponds to the recurrence $T(n) = 4T(\frac{n}{2}) + \Theta(1)$

\hspace*{1mm}$T(\frac{n}{2})$ \hspace*{12mm}$\Theta(1)$

\hspace*{2mm}$\frac{n}{4}$\hspace*{5mm}$\frac{n}{4}$ \hspace*{8mm}$2\Theta(1)$

$\frac{n}{8}$ $\frac{n}{8}$ $\frac{n}{8}$ $\frac{n}{8}$ \hspace*{6mm}$4\Theta(1)$

and so on...

Height: $2^n\Theta(1)$

Total Complexity: $\sum_{i=1}^{n}\Theta(1)$

At each level it most closely match a multiple of $\Theta(1)$

\section*{Problem 4}

Find a recurrence that describes the worst-case complexity of the Third-Sort algorithm below.  Show all work.
You may assume that the floor function ($\lfloor$ $\rfloor$) takes constant time.

\fbox {
    \begin{minipage}{\textwidth}
        \paragraph{Input}: data: array of integers
        \paragraph{Input}: n: the length of data \\
        \textbf{Output} a permutation of data such that $data[1] \leq data[2] \leq ... \leq data[n]$ \\
        \textbf{c} 01 \textbf{Algorithm}: ThirdSort \\
        \textbf{c} 02 \textbf{if} $n=1$ \textbf{then} \\
        \textbf{c} 03  \hspace*{4mm}\textbf{return} $data$ \\
        \textbf{c} 04 \textbf{else if} $n=2$ \textbf{then} \\
        \textbf{c} 05  \hspace*{4mm}\textbf{if} $data[1] > data[2]$ \textbf{then} \\ 
        \textbf{c} 06      \hspace*{8mm}Swap $data[1]$ and $data[2]$ \\
        \textbf{c} 07  \hspace*{4mm}\textbf{end} \\
        \textbf{c} 08  \hspace*{4mm}\textbf{return} data \\
        \textbf{c} 09 \textbf{else} \\
        \textbf{c} 10  \hspace*{4mm}$third = \lfloor\frac{n}{3}\rfloor$ \\
        \textbf{T}$(\frac{2n}{3})$ 11  \hspace*{4mm} Call ThirdSort on $data[1..n-third]$ \\
        \textbf{T}$(\frac{2n}{3})$ 12  \hspace*{4mm} Call ThirdSort on $data[third+1..n]$ \\
        \textbf{T}$(\frac{2n}{3})$ 13  \hspace*{4mm} Call ThirdSort on $data[1..n-third]$ \\
        \textbf{c} 14  \hspace*{4mm} \textbf{return} $data$ \\
        \textbf{c} 15 \textbf{end}
    \end{minipage}
}

$T(n) = 3T(\frac{2n}{3})) + f(n)$ which in this case f(n) occurs $n$ times which is $\Theta(lg(n))$

\textbf{Answer} $T(n) = 3\Theta(lg(n)) + \Theta(lg(n)))$

\section*{Problem 5}

Use the Master Theorem to find the worst-case complexity of ThirdSort.  You may assume that $f(n)$ is
regular if relevant.  Recall that $log_a(b) = \frac{ln(b)}{ln(a)}$ (you may need a calculator for this one).
Be sure to include the value of $c$ and the case of the Master Theorem in your answer.

\noindent
$T(n) = aT(\frac{n}{b}) + f(n)$

\noindent
$\frac{n}{b}$ is the size of recursive calls

\noindent
$a$ is the  number of recursive calls, possibly equal to $b$ 

\noindent
$f(n)$ is the time for other code

\noindent
From Problem 4, $T(n) = 3\Theta(lg(n)) + \Theta(lg(n)) = 4\Theta(lg(n))$, which means $a = 4$ and $b = 2$

\noindent
\textbf{Answer} 

\noindent
$c = log_b(a) = \frac{ln(b)}{ln(a)} = log_2(4) = \frac{ln(4)}{ln(2)} = \frac{2}{1} = 2$

\noindent
Given that $n^c = n^2$, and it is greater than $\Theta(lg(n))$, $T(n)$ is $\Theta(n^c)$

\end{document}

