\documentclass[12pt, letterpaper]{article}
% Custom Commands for User Info
\newcommand{\studentname}          {John Rizzo           }
\newcommand{\classname}            {CS590-A Algorithms   }
\newcommand{\professorname}        {Dr. William Hendrix  }
\newcommand{\assignmentdescription}{Homework 1 Algorithms}
\newcommand{\duedate}              {January 31, 2025     }

\title{\classname \\ \assignmentdescription}
\author{\studentname}
\date{\duedate}

% Packages
\usepackage[utf8]{inputenc}
\usepackage{amsmath}
\usepackage{amssymb}
\usepackage{geometry}
\geometry{margin=1in}
\usepackage{fancyhdr}
\usepackage{datetime}
\usepackage{pgfplots}

% Header and Footer setup
\renewcommand{\headrulewidth}{0.4pt}
\renewcommand{\footrulewidth}{0.4pt}
\setlength{\headheight}{14.49998pt}
\addtolength{\topmargin}{-2.49998pt}

\pgfplotsset{width=10cm,compat=1.18}

% Document Start
\begin{document}

\noindent
\normalsize \textbf{Student:     \studentname} \\ [5pt]
            \textbf{Course:      \classname} \\ [5pt]
            \textbf{Instructor:  \professorname} \\ [5pt]
            \textbf{Due Date:    \duedate} \\ [5pt]
            \textbf{Description: \assignmentdescription}

\vspace{0.5cm}

% Problem Sections
\section*{Problem 1}
\textbf{Definition 1.} $a \mid b$ \textit{("a divides b") if and only if there exists some integer $k$ such that $b = ak$. Equivalently, $a \mid b$ if and only if $b$ has a remainder of $0$ when divided by $a$ (see question 2).}

\begin{enumerate}
    \item Using the formal definition of divisibility above, prove that there exist positive integers $a$, $b$, and $c$ such that $a \mid bc$, but $a \nmid b$ and $a \nmid c$.

    \item[] \textbf{Proof.} Let $a = 6$, $b = 2$, and $c = 3$. Then $bc = 6$ and $a \mid bc$ since $6 = 6 \cdot 1$. However, $a \nmid b$ and $a \nmid c$ since $2 = 6 \cdot 0 + 2$ and $3 = 6 \cdot 0 + 3$. $\blacksquare$

    \item[] \textbf{Theorem 1}. \textit{The Division Algorithm. For any integers $a$ and $b$ where $b \neq 0$, there exist a unique pair of integers $q$ and $r$ such that $a = qb + r$ and $0 \leq r \leq b$.  The integers $q$ and $r$ are known as the quotient and remainder of $a \div b$, respectively.}

    \item Using the formal definition of the remainder above, prove that if $n$ and $m$ are positive integers such that $n$ has a remainder of $r$ when divided by $m$ and $r < \sqrt{m}$, $n^2$ has a remainder of $r^2$ when divided by $m$.

    \item[] \textbf{Proof.} Let $n$, and $m$ be positive integers such that $n \div m$ has a remainder $r$.
    
    By Theorem 1, $m = qn + r$ and $m = qn^2 + r^2$.  
    
    $\sqrt{m} = \sqrt{qn^2 + r^2}$
    
    $\sqrt{m} = \sqrt{qn^2} + \sqrt{r^2}$
    
    $\sqrt{m} = n\sqrt{q} + r$.
        
    \item Use the formal definition of Big-Oh to prove that if $f(n) = n^x + an^y$, where $a$, $x$, and $y$ are positive integers such that $x > y$, $f(n) = O(n^x)$.

    \textit{Big-Oh is defined as $f(n) = O(g(n))$ if there exists a positive constant $c$ and a positive integer $n_0$ such that $f(n) \leq cg(n)$ for all $n \geq n_0$.}

    \textbf{Solution}. In this case, $f(n) = n^x + an^y$ and $g(n) = n^x$. Let us assume y = x. Then $f(n) = n^x + an^x$ which can be reduced to $f(n) = n^x(1 + a)$.  Let $1+a = c$ and $n_0 = 1$ to establish the upper bound. Then $f(n) = n^x(1 + a) \leq cn^x$ for all $n \geq n_0$. Therefore, $f(n) = O(n^x)$. $\blacksquare$

    \item Use the formal definition of Big-Omega to prove that if $f_1(n)$, $f_2(n)$, $g_1(n)$, and $g_2(n)$ are functions such that $f_1(n) = \Omega(g_1(n))$ and $f_2(n) = \Omega(g_2(n))$, $f_1(n) + f_2(n) = \Omega(\max(g_1(n), g_2(n)))$.

    \textbf{Solution}. Big Omega is defined as $f(n) = \Omega(g(n))$ if there exists a positive constant $c$ and a positive integer $n_0$ such that $f(n) \geq cg(n)$ for all $n \geq n_0$.

    For $f_1(n)$, $n \geq c_1n$ for say $n_0=0$ and $c_1=0.5$. The same is true for $f_2(n)$ and $g_2(n)$  All of the functions are separated by a constant factor.  Given that $f_1(n)$ and $f_2(n)$ were both larger than $g_1(n)$ and $g_2(n)$ the sum of the two must be larger than the maximum of either $g_1(n)$ or $g_2(n)$.$\blacksquare$
\end{enumerate}

\end{document}
